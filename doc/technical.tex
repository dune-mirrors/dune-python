\documentclass[11pt]{article}

\usepackage{amsmath}
\usepackage{url}
\usepackage{listings}
\lstset{language=C++, basicstyle=\small\ttfamily,
  stringstyle=\ttfamily, commentstyle=\it, extendedchars=true}

\title{Technical Documentation of \textbf{dune-python}}

\author{
Dominic Kempf$^\dagger$
}

\date{\today}

\begin{document}

\maketitle
\tableofcontents
\pagebreak

\section{Introduction}
dune-python is a Dune module that aims at providing the infrastructure needed whenever a Dune module does also contains some python code.

\section{Structure of a dune module containing python code}
The structure of a Dune module follows some conventions on where to place code. This sections aims at extending those good practices to python code. \\

\begin{itemize}
 \item All python code should be placed in a top-level directory called \lstinline!python!.
 \item The contents of the directory \lstinline!python! should be a \lstinline!pip!-installable package\footnote{For an introduction into \lstinline!pip!, see }.
\end{itemize}

\section{The Dune virtualenv}

\section{The python related CMake macros}

\section{Summary - A checklist for people that developped a dune module that contains python code}


\end{document}
